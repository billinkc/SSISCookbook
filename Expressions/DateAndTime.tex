\chapter{Date and time}
We often need to generate a date string in various formats. A common challenge encountered is getting that format just right.
Unless you have reasons to the contrary, using a format of year, month and date, all numeric, results in a sortable date which 
makes it easier to find specific files in large collections.

\paragraph{Getting a date}

There are different mechanisms for acquiring a date. A popular approach is to use \verb+GETDATE()+ \href{http://msdn.microsoft.com/en-us/library/ms139875.aspx}{GETDATE(SSIS Expression)}

It's built into the SSIS Expression language but I find that the System scoped variable, \verb+StartTime+ provides a better reference point as it is evaulated only when the package begins. This is in contrast to GETDATE which is evaluated each time it is inspected. For a long running package, the drift in values may not be acceptable. It might also be \begin{em}exactly\end{em} what is desired. Assume we have a Derived Column transformation in a package which adds a column named InsertDate into our buffer. 

Consider the following two queries, which would more correctly describe the data loaded from today's data load?

Quote
\begin{quote}
SELECT 
    T.*
    
FROM 
    MyTable AS T

WHERE 
    T.InsertDate = '2015-01-17T12:24:48.123'; 
\end{quote}

VerbatimInput method

\VerbatimInput{Expressions/InsertDate.Scan.sql}

verbatiminput method

\verbatiminput{Expressions/InsertDate.Scan.sql}

\section{YYYY}
This expression defines how to get a 4 digit year. This is sometimes referred to as CCYY (century, century, year, year). The expression simply applies the \verb|YEAR()| function to a variable which returns an integer value. 

\verbatiminput{Expressions/YYYY.exp}

To make use of it, it's been my experience that we will be using this value for string concatenation instead of mathematical operations: addition, subtraction, etc. Therefore, we explicitly cast the result of the \verb|YEAR()| as a unicode string, \verb|DT_WSTR| of length 4

Biml
\verbatiminput{Expressions/YYYY.biml}


\section{YY}
This expression defines how to get a 2 digit year. We apply the \verb|RIGHT()| function and get the last 2 digits. There is no need to explicitly cast to a string as it is handled by the function call.

\verbatiminput{Expressions/YY.exp}

Biml
\verbatiminput{Expressions/YY.biml}

\section{MM}
This expression defines how to get a 2 digit month such that March is 03 instead of 3. The leading zero is important for character based sorting. We will use the \verb|MONTH()| function which returns an integer from 1 to 12. We convert that to a string, prepend a zero to it and then shear off the last two characters using the \verb|RIGHT()| function. The trimming is only required for October, November, and December but the logic is cleaner to unconditionally apply \verb|RIGHT()|.

\verbatiminput{Expressions/MM.exp}

Biml
\verbatiminput{Expressions/MM.biml}

\section{DD}
This expression defines how to get a 2 digit day such that instead of 9 we get 09. It is the same process used in MM except with a call to  \verb|DAY()|. We continue to prepend a zero to our value and shear off the last two digits.

\verbatiminput{Expressions/DD.exp}

Biml
\verbatiminput{Expressions/DD.biml}

\section{YYYY-MM-DD}
The most common format is a year, month and day separated by a dash. This recipe uses what we've build in the YYYY, MM and DD sections.


verbatimimport 

\verbatiminput{Expressions/YYYY-MM-DD.exp}


Biml
\verbatiminput{Expressions/YYYY-MM-DD.biml}




\section{YYYYMMDD}
A more compact version of the year month date string is represented below. 

\verbatiminput{Expressions/YYYYMMDD.exp}


Biml
\verbatiminput{Expressions/YYYYMMDD.biml}